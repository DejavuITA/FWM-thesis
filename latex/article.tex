\documentclass[12pt,a4paper,twocolumn,twoside]{article}
\usepackage[utf8]{inputenc}
\usepackage[english]{babel}
\usepackage{amsmath}
\usepackage{amsfonts}
\usepackage{amssymb}
\usepackage{graphicx}
\usepackage{parcolumns}
\usepackage[left=2.cm,right=2.cm,top=2.cm,bottom=2.cm]{geometry}
\usepackage[final]{pdfpages} %serve per aggiungere altre pagine pdf al file
\usepackage{siunitx}

\author{Davide Bazzanella}
\title{Improving phase-matching of FWM with temperature}

\begin{document}
\includepdf[pages={1}]{intestazione.pdf} % \documentclass[11pt,a4paper]{article}
\usepackage[utf8]{inputenc}
\usepackage[english]{babel}
\usepackage{amsmath}
\usepackage{amsfonts}
\usepackage{amssymb}
\usepackage{graphicx}
\usepackage{parcolumns}
\usepackage[left=0.5cm,right=.5cm,top=.5cm,bottom=.5cm]{geometry}

\begin{document}

\begin{titlepage}
\begin{center}

\includegraphics[width=0.7\textwidth]{unitn_logo.png}~\\[1.2cm]

\hrule
$$$$
$$$$

\textsc{\LARGE \textbf{DEPARTMENT OF PHYSICS}}\\[0.5cm]
\textsc{\LARGE \textbf{DEGREE IN PHYSICS – LAUREA IN FISICA}}\\[2.5cm]
\textsc{\Large THESIS}\\[1.3cm]

{ \huge \bfseries Thermal tuning of phase-matching}\\
[0.5cm]
{ \huge \bfseries in a multimodal SOI waveguide}\\
[0.5cm]
{ \huge \bfseries with $\chi^{(3)}$ non linearity}\\
[3.0cm]

% ATTENZIONE! Ha bisogno del pacchetto parcolumns per funzionare
\begin{parcolumns}{2}
   \colchunk{\Large Thesis Advisor:\\ \LARGE \textbf{Mattia Mancinelli}}
   \colchunk[2]{\Large Student:\\ \LARGE \textbf{Davide Bazzanella}}
\end{parcolumns}
$$$$
$$$$
$$$$
$$$$
$$$$
$$$$
$$$$
$$$$

\large ACADEMIC YEAR 2014/2015
\hrule
\vfill
\textbf{Abstract}: study of degenerate Four-Wave Mixing fenomena in a rectangular silicon-based waveguide.
Analysis of its behaviour at different temperatures to maximize phase-match.
Developement of a project to build an actual prototype.

\vfill

{\large}

\end{center}
\end{titlepage}

\end{document}

\cleardoublepage 
\tableofcontents

\cleardoublepage 
\section{Introduction}
\subsection{Motivation}
\subsection{Partially degenerate Four-Wave Mixing}
Four-Wave Mixing (FWM) is a nonlinear optical effect

****
\section{Linear optics}

A linear dielectric medium is characterized by a linear relation between the polarization density and the electric field, $\textbf{P} = \varepsilon_0 \chi \textbf{E}$, where $\varepsilon_0$ is the permittivity of free space and $\chi$ is the susceptibility of the medium.

****

In linear optics the technology for trasmitting a light wave by confining it in a finite space is the guided-wave optics.
The instruments employed for achieving such purpose are called waveguides.
\subsection{Silicon waveguides}
Silicon waveguides are dielectric waveguides and make use of the interface between two media with different refractive index.
Precisely they exploit the phenomenon of total internal reflection: if a propagating wave reaches a boundary between two mediums, one with higher refractive index ($n_H$) and another with lower refractive index ($n_L$), with an angle of incidence greater than the \textit{critical angle}, then the wave is completely reflected in the first medium.
%accorciare la precedente frase%
Angles are defined relative to the normal vector of the interface and the critical angle can be obtained by the Snell law:

$$	\theta_C = \arcsin \left( \frac{n_L}{n_H} \right)$$

Silicon waveguides are composed by core and cladding \textbf{(?)}. %is called cladding only for fiber or also for slab/strip wg?
The former is the higher refractive index medium in which the wave is confined, the latter is the lower refractive index medium which surrounds the core, thus creating the interface. Theoretically the cladding could also be vacuum.

Considering an ideal dielectric waveguide (no absorption) and the total field distribution as sum of transverse electromagnetic (TEM) plane waves is possible to study the propagation of the wave along the waveguide with an electromagnetic analysis.
Furthermore it can be shown that not all field distribution are transmitted through the waveguide without energy loss \textbf{(?)}.

% Modes are fields that maintain the same transverse distribution and polarization at all locations along the waveguide axis.
Those field distributions, which maintain the same transverse distribution and polarization at all locations along the waveguide axis, are called modes.
These modes can be categorized in two main groups, depending on the polarization of the wave: transverse electric (TE) modes and transverse magnetic (TM) modes, which have respectively the electric and magnetic field transverse to the waveguide axis.
Both TE and TM modes can be again classified by their order, that is proportional to the number of maxima of the modulus of the electric and magnetic fields in the core section.


\subsection{Rectangular dielectric waveguides}
In a rectangular dielectric waveguide the classification of modes need two indexes because we have two degrees of freedom.

\section{Non-linear optics}
A nonlinear dielectric medium is characterized by a nonlinear relation between P and E, such as
$$\textbf{P} = \varepsilon_0 \left( \chi^{(1)} \cdot \textbf{E} + \chi^{(2)} : \textbf{E}^2 + \chi^{(3)} \vdots \textbf{E}^3 + \cdots \right) $$
In everyday condition linear effects are much larger than nonlinear ones.
The relation between $\mathrm{P}$ and $\mathrm{E}$ becomes nonlinear when $\mathrm{E}$ has a value comparable to interatomic electric fields, which are typically $\sim 10^5-10^8$ \si{\V\per\m}.

For an isotropic medium the second order term is zero (in the dipole approximation) and this is our scenario.
Thus the dominant nonlinearity is the third order and the material is called a Kerr medium.
\subsection{Third order effect: Four-Wave Mixing}
Many processes are result of third order nonlinearities: Third-Harmonic Generation (THG), Optical Kerr Effect, Cross-phase modulation (XPM), Self-phase modulation (SPM) and Four-Wave Mixing (FWM).

Four-Wave Mixing originates from the third order nonlinear response of a material to an electromagnetic field, precisely from the behaviour of bound electrons.
This parametric process involve nonlinear interaction among four optical waves.
\section{Simulating and coding ?}

$$\omega_1 + \omega_2 = \omega_3 + \omega_4$$
$$\omega_1 = \omega_2 = \omega_P$$
$$\omega_I = 2\omega_P - \omega_S$$
$$\Omega_S = \omega_1 + \omega_3 = \omega_4 - \omega_1$$
where we assume $\omega_3 < \omega_4$.
$$\Omega_S = |\omega_P - \omega_I| = |\omega_S - \omega_P|$$

$$\Delta k = \beta_3 + \beta_4 - \beta_1 - \beta_2$$
$$\Delta k = \beta_3 + \beta_4 - 2\beta_1$$

$$L_{coh} = \frac{2\pi}{|\Delta k|}$$



\section{Prototyping}
\subsection{Heater configurations}
\subsection{Different geometries}

\newpage
\begin{thebibliography}{9}
\bibitem{agrawal} G. Agrawal. Nonlinear Fiber Optics, 5th edition. 2013
\bibitem{saleh} B. E. A. Saleh, M. C. Teich. Fundamentals of photonics, 2nd edition. 2007
\end{thebibliography}
\end{document}
% Ref. \cite{agrawal}